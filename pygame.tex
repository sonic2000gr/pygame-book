% Main file for compiling the entire book
% Games in Python and Pygame
% 
% (C) 2011-2012 Manolis Kiagias
% Published under the Creative Commons License
%
%!TEX TS-program = xelatex
%!TEX encoding = UTF-8 Unicode
%
% Please note: This book requires XeLaTeX
% and Python / Pygments to compile.
% Use easy_install to get Pygments for your OS, e.g:
% sudo easy_install Pygments
% The minted package should also be installed in your TeX distribution
%
% This book uses the free Liberation family fonts
% You may have to add them to your OS fonts or alter
% the setfont statements in this file to suitable
% alternatives
%
% Version 1.0 - September 2012
%
\documentclass[a4paper,twoside,12pt,landscape]{book}
\usepackage{fontspec,xltxtra,xunicode}
% Begin new paragraphs with an empty line rather than an indent
\usepackage[parfill]{parskip}
\usepackage[landscape, hmargin={1in, 1.5in}, vmargin={1in,1in}]{geometry}
\usepackage{xgreek}
\usepackage{multirow}
\usepackage{fancyvrb}
\usepackage{minted}
\usepackage{sidecap}
\usepackage{xcolor}
\usepackage[colorlinks=true,linkcolor=black]{hyperref}
% Scaling based on Scale=MatchLowercase and Times New Roman
% causes inconsistent output between OS X and FreeBSD
% Therefore Scale is now set as an absolute number
% Times New Roman is not used anywhere
\defaultfontfeatures{Scale=0.975,Mapping=tex-text}
%\setromanfont{Times New Roman}
\setsansfont{Liberation Sans}
\setmonofont{Liberation Mono}
\setmainfont{Liberation Serif}
\usepackage{multicol}
\usepackage{graphicx}
\setcounter{secnumdepth}{4}
\setcounter{tocdepth}{4}
\usepackage{fancyhdr}
\pagestyle{fancy}
\renewcommand{\chaptermark}[1]{%
	\markboth{#1}{}}
\renewcommand{\sectionmark}[1]{%
	\markright{\thesection\ #1}}
\fancyhf{}
\fancyhead[LE,RO]{\bfseries\thepage}
\fancyhead[LO]{\bfseries\rightmark}
\fancyhead[RE]{\bfseries\leftmark}
\renewcommand{\headrulewidth}{0.5pt}
\addtolength{\headheight}{2pt}
\renewcommand{\footrulewidth}{0pt}
\addtolength{\headheight}{0.5pt}
\fancypagestyle{plain}{%
	\fancyhead{}
	\renewcommand{\headrulewidth}{0pt}}
\renewcommand{\figurename}{Εικόνα}
%
% User defined environments
%
\newenvironment{inthebox}{\line(1,0){390}\\} %
    {\line(1,0){390}}
\DefineVerbatimEnvironment{richverb}%
	{Verbatim}{commandchars=\|\[\], commentchar=\!}
\definecolor {bg} {rgb}{0.98, 0.98, 0.98}
\sidecaptionvpos{figure}{c}
%
%
\author{Μανώλης Κιαγιάς, MSc}
\title{Παιχνίδια σε Python και Pygame}
\date{01/09/2012}
\begin{document}
\frontmatter
%
% Book intro pages (frontmatter)
%
\maketitle
\begin{center}
Κάθε γνήσιο αντίτυπο φέρει την υπογραφή του συγγραφέα:
\begin{tabular}{p{0.9\textwidth}}
\\
\\
\\
\end{tabular}

\bigskip
Έκδοση: 1η -- Χανιά, 01/09/2012

[ Αριθμός αντιτύπου: Web Edition ]

\bigskip
Copyright \copyright 2011 -- 2012 Μανώλης Κιαγιάς

Το Έργο αυτό διατίθεται υπό τους όρους της Άδειας:\\
\includegraphics[scale=0.2]{images/license/cc-logo}\\
\textbf{Αναφορά -- Μη Εμπορική Χρήση --  Παρόμοια Διανομή 3.0 Ελλάδα}\\
Μπορείτε να δείτε το πλήρες κείμενο της άδειας στην τοποθεσία:\\
\url{http://creativecommons.org/licenses/by-nc-sa/3.0/gr/}
\end{center}
\newpage
\subsection*{Είναι Ελεύθερη:}

\noindent
\textbf{Η Διανομή} -- Η αναπαραγωγή, διανομή, μετάδοση και παρουσίαση του Έργου σε κοινό

\subsection*{Υπό τις ακόλουθες προϋποθέσεις:}

\vspace{1em}
\noindent
\parbox{1.5cm}{\includegraphics[scale=1.5]{images/license/cc_by_30}}
\parbox{10.5cm}{\textbf{Αναφορά Προέλευσης} — Θα πρέπει να αναγνωρίσετε την προέλευση στο έργο σας με τον τρόπο που έχει ορίσει ο δημιουργός του ή το πρόσωπο που σας χορήγησε την άδεια (χωρίς όμως να αφήσετε να εννοηθεί  ότι εγκρίνουν  με οποιονδήποτε τρόπο εσάς ή τη χρήση του έργου από εσάς).}

\vspace{1em}
\noindent
\parbox{1.5cm}{\includegraphics[scale=1.5]{images/license/cc_nc_30}}
\parbox{10.5cm}{\textbf{Μη Εμπορική Χρήση} --  Δεν μπορείτε να χρησιμοποιήσετε αυτό το έργο για εμπορικούς σκοπούς.}

\vspace{1em}
\noindent
\parbox{1.5cm}{\includegraphics[scale=1.5]{images/license/cc_sa_30}}
\parbox{10.5cm}{\textbf{Παρόμοια Διανομή}  — Αν αλλοιώσετε, τροποποιήσετε ή δημιουργήσετε κάποιο παράγωγο έργο το οποίο βασίζεται στο παρόν έργο, μπορείτε να διανείμετε το αποτέλεσμα μόνο με την ίδια ή παρόμοια με αυτή άδεια.}

\subsection*{Με την κατανόηση ότι:}

\noindent
\textbf{Αποποίηση} -- Οποιεσδήποτε από τις παραπάνω συνθήκες μπορούν να παρακαμφθούν αν πάρετε την άδεια του δημιουργού ή κατόχου των πνευματικών δικαιωμάτων.

\vspace{1em}
\noindent
\textbf{Άλλα Δικαιώματα} -- Σε καμιά περίπτωση τα ακόλουθα δικαιώματα σας, δεν επηρεάζονται από την Άδεια:

\begin{itemize}
  \item Η δίκαιη χρήση και αντιμετώπιση του έργου
  \item Τα ηθικά δικαιώματα του συγγραφέα
  \item Τα ενδεχόμενα επί του έργου δικαιώματα τρίτων προσώπων, σχετικά με τη χρήση του έργου, όπως για παράδειγμα η δημοσιότητα ή ιδιωτικότητα.
\end{itemize}

\vspace{1em}
\noindent
\textbf{Σημείωση} -- Για κάθε επαναχρησιμοποίηση ή διανομή, πρέπει να καταστήσετε σαφείς στους άλλους τους όρους της άδειας αυτού του Έργου. Ο καλύτερος τρόπος να το πράξετε αυτό, είναι να δημιουργήσετε ένα σύνδεσμο με το διαδικτυακό τόπο της παρούσας άδειας:
\begin{center}
\url{http://creativecommons.org/licenses/by-nc-sa/3.0/gr/}
\end{center}
\line(1,0){665}\\\\
\begin{tabular}{p{0.95\textwidth}}
Το βιβλίο αυτό στοιχειοθετήθηκε σε \XeLaTeX{}.  Ο πηγαίος κώδικας του είναι διαθέσιμος στις δικτυακές τοποθεσίες που αναφέρονται παρακάτω και μέσω mercurial repository.\\
\end{tabular}

\bigskip
Επισκεφθείτε το δικτυακό τόπο του μαθήματος
και κατεβάστε την τελευταία έκδοση του βιβλίου και διορθώσεις:

\bigskip
\begin{center}
\url{http://pygamegr.wordpress.com}
\end{center}
Στην παραπάνω τοποθεσία θα βρείτε και όλα τα προγράμματα που περιέχονται στο
παρόν βιβλίο καθώς και τα συνοδευτικά αρχεία (εικόνες και ήχους) που απαιτούνται
για την εκτέλεση τους.
\bigskip
Σε περίπτωση προβλήματος χρησιμοποιήστε το mirror site:
\bigskip
\begin{center}
\url{http://www.freebsdworld.gr/files/python-pygame.pdf}
\end{center}
\newpage
\vspace*{0.8in}
Αφιερώνεται στους μαθητές μου Κοντορίνη Ανδρέα και Τζωρτζάκη Εύα που
υπομονετικά άντεξαν εμένα, την python και το pygame για μια ολόκληρη χρονιά!
\newpage
(Κενή Σελίδα)
\newpage
\section*{Πρόλογος}
{\em Οι υπολογιστές έχουν αλλάξει δραματικά από τη δεκαετία του 80}

Δυστυχώς η περίφημη ``ευκολία χρήσης'' των σημερινών υπολογιστών έφερε μαζί της και ένα ανεπιθύμητο αποτέλεσμα: το τέλος της {\em δημιουργικής} χρήσης. Έχετε σκεφτεί που και πως χρησιμοποιούνται οι υπολογιστές σήμερα;
%
\begin{itemize}
\item[-] Για πρόσβαση στο Internet (και ειδικά στα social media)
\item[-] Για παιχνίδια
\item[-] Για μουντές εργασίες γραφείου
\end{itemize}
%
Κάποτε όμως ο ταπεινός ``χρήστης'' του υπολογιστή ήταν και ο προγραμματιστής
του. Στη δεκαετία του 80 δεν φοβόμασταν τις εντολές, το αντίθετο μάλιστα --
τις επιζητούσαμε.  Ο έλεγχος του ανθρώπου στη μηχανή, η δυνατότητα να
προγραμματίσεις το μηχάνημα να κάνει κάτι πρωτότυπο, είτε υπολογισμό, είτε παιχνίδι, ήχο, γραφικά ήταν η απόλυτη πρόκληση.

Μερικά χρόνια μετά και ο περίφημος ``μέσος χρήστης'' χρησιμοποιείται πλέον
από το μηχάνημα αντί να το χρησιμοποιεί ο ίδιος. Οι υπολογιστές παίζουν
με τα νεύρα του αντί να παίζει αυτός με τις δυνατότητες τους.
Ταυτόχρονα έχει αγοράσει το παραμύθι ότι ο υπολογιστής είναι μια
καταναλωτική συσκευή όπως η τηλεόραση ή το ψυγείο του -- και άρα είναι
εύκολος στη χρήση.

Θα σας διαψεύσω: ο υπολογιστής γενικής χρήσης δεν ήταν ποτέ εύκολος στη
χρήση -- και ευτυχώς ούτε θα γίνει. Γιατί αν γίνει, δεν θα είναι υπολογιστής
(βλέπε iPad). Αν μια πένσα είναι ένα εργαλείο που μεγεθύνει τη μυϊκή
δύναμη, ο υπολογιστής είναι το αντίστοιχο εργαλείο για το μυαλό. Και το μυαλό είναι το δυσκολότερο σε χρήση εργαλείο του ανθρώπου έτσι και αλλιώς.

Για να γίνει ξανά ο υπολογιστής σας δικός σας υπηρέτης και όχι το αντίθετο, πρέπει να μάθετε να τον προγραμματίζετε. Και αυτό αποσκοπούμε μέσα από αυτό το βιβλίο -- αλλά με ένα τρόπο ευχάριστο και αποφεύγοντας την πεπατημένη: με προγραμματισμό παιχνιδιών!

Ταυτόχρονα το βιβλίο αυτό είναι μια αναδρομή στη χρυσή εποχή των οικιακών υπολογιστών και των προγραμμάτων που κάποιοι από εμάς γράφαμε στην εφηβεία μας.

Τα ``Παιχνίδια σε Python και Pygame'' ξεκίνησαν ως μια σειρά άρθρων στο περιοδικό \url{http://www.deltahacker.gr} -- αλλά ήταν από την πρώτη στιγμή σίγουρο
ότι θα κυκλοφορούσαν και ως βιβλίο.

Γιατί, πολύ απλά, {\em το βιβλίο αυτό το χρώσταγα στον εαυτό μου}.
\smallskip
\begin{flushright}
Μανώλης Κιαγιάς, Σεπτέμβριος 2012
\end{flushright}
\newpage
\tableofcontents
\listoffigures

\mainmatter
%
% Chapter 1 section collection file
%
\input chapters/chapter1/section1.tex

%
% Chapter 2 section collection file
%
\input chapters/chapter2/section1.tex
\input chapters/chapter2/section2.tex

%
% Chapter 3 section collection file
%
\input chapters/chapter3/section1.tex

%
% Chapter 4 section collection file
%
\input chapters/chapter4/section1.tex
\input chapters/chapter4/section2.tex

%
% Chapter 5 section collection file
%
\input chapters/chapter5/section1.tex

%
% Chapter 6 section collection file
%
\input chapters/chapter6/section1.tex

%
% Chapter 7 section collection file
%
\input chapters/chapter7/section1.tex

%
% Chapter 8 section collection file
%
\input chapters/chapter8/section1.tex
\input chapters/chapter8/section2.tex

%
% Chapter 9 section collection file
%
\input chapters/chapter9/section1.tex
\input chapters/chapter9/section2.tex

\appendix
\chapter{Προγράμματα}
\section{Guess the Number}
\label{listing:guess}
\VerbatimInput{sources/1/guess1.py}
\section{The Adventure}
\label{listing:adventure}
\VerbatimInput{sources/2/adventure.py}
\section{Hello Pygame}
\label{listing:hello-pygame}
\VerbatimInput{sources/4/hello-pygame.py}
\section{Colorbars}
\label{listing:colorbars}
\VerbatimInput{sources/4/colorbars.py}
\section{Bouncing Ball}
\label{listing:bouncing-ball}
\VerbatimInput{sources/4/bouncing-simple.py}
\section{Αντικειμενοστραφές Bouncing Ball}
\label{listing:bouncing-ball-oop}
\VerbatimInput{sources/5/bouncing-oop.py}
\section{Graphics Match}
\label{listing:graphics-match}
\VerbatimInput{sources/6/graphics-match.py}
\section{Pygame Invaders}
\label{listing:pygame-invaders}
\VerbatimInput{sources/9/pygame-invaders.py}

\end{document}
